%%% Для сборки выполнить 2 раза команду: pdflatex <имя файла>

\documentclass[a4paper,12pt]{article}

\usepackage{ucs}
\usepackage[utf8x]{inputenc}
\usepackage[russian]{babel}
%\usepackage{cmlgc}
\usepackage{graphicx}
\usepackage{listings}
\usepackage{xcolor}
\usepackage{titlesec}
%\usepackage{courier}

\makeatletter
\renewcommand\@biblabel[1]{#1.}
\makeatother

\newcommand{\myrule}[1]{\rule{#1}{0.4pt}}
\newcommand{\sign}[2][~]{{\small\myrule{#2}\\[-0.7em]\makebox[#2]{\it #1}}}

% Поля
\usepackage[top=20mm, left=30mm, right=10mm, bottom=20mm, nohead]{geometry}
\usepackage{indentfirst}

% Межстрочный интервал
\renewcommand{\baselinestretch}{1.50}

% ------------------------------------------------------------------------------
% minted
% ------------------------------------------------------------------------------
\usepackage{minted}


% ------------------------------------------------------------------------------
% tcolorbox / tcblisting
% ------------------------------------------------------------------------------
\usepackage{xcolor}
\definecolor{codecolor}{HTML}{FFC300}

\usepackage{tcolorbox}
\tcbuselibrary{most,listingsutf8,minted}

\tcbset{tcbox width=auto,left=1mm,top=1mm,bottom=1mm,
right=1mm,boxsep=1mm,middle=1pt}

\newtcblisting{myr}[1]{colback=codecolor!5,colframe=codecolor!80!black,listing only, 
minted options={numbers=left, style=tcblatex,fontsize=\tiny,breaklines,autogobble,linenos,numbersep=3mm},
left=5mm,enhanced,
title=#1, fonttitle=\bfseries,
listing engine=minted,minted language=r}

%%%%%%%%%%%%%%%%%%%%%%%%%%%%%%%%%%%%%%%

\begin{document}

%%%%%%%%%%%%%%%%%%%%%%%%%%%%%%%
%%%                         %%%
%%% Начало титульного листа %%%

\thispagestyle{empty}
\begin{center}


\renewcommand{\baselinestretch}{1}
{\large
{\sc Петрозаводский государственный университет\\
Институт математики и информационных технологий\\
Кафедра информатики и математического обеспечения
}
}

\end{center}


\begin{center}
%%%%%%%%%%%%%%%%%%%%%%%%%
%
% Раскомментируйте (уберите знак процента в начале строки)
% для одной из строк типа направления  - бакалавриат/
% магистратура и для одной из
% строк Вашего направление подготовки
%
 %Направление подготовки бакалавриата \\
% 01.03.02 --- Прикладная математика и информатика \\
% 09.03.02 --- Информационные системы и технологии \\
09.03.04 --- Программная инженерия \\
%%%%%%%%%%%%%%%%%%%%%%%%%
\end{center}

\vfill

\begin{center}
{\normalsize 
	Отчет о проектной работе по курсу <<Основы информатики и программирования>>}

\medskip

%%% Название работы %%%
	{\Large \sc {Приложение <<2048>> }} \\
\end{center}

\medskip

\begin{flushright}
\parbox{11cm}{%
\renewcommand{\baselinestretch}{1.2}
\normalsize
	Выполнил:\\
% Выполнили:\\
%%% ФИО студента %%%
студент 1 курса группы 22107
\begin{flushright}
	Д. С. Мельников \sign[подпись]{4cm}
\end{flushright}

Руководитель:\\
А. В. Бородин, старший преподаватель
%%% Второй участник %%%
% студента 1 курса группы 2210X
% \begin{flushright}
% 	И. О. Фамилия \sign[подпись]{4cm}
% \end{flushright}

%%%%%%%%%%%%%%%%%%%%%%%%%
% девушкам применять "Выполнила" и "студентка"
%%%%%%%%%%%%%%%%%%%%%%%%%
}
\end{flushright}

\vfill

\begin{center}
\large
    Петрозаводск --- 2021
\end{center}

%%% Конец титульного листа  %%%
%%%                         %%%
%%%%%%%%%%%%%%%%%%%%%%%%%%%%%%%

%%%%%%%%%%%%%%%%%%%%%%%%%%%%%%%%
%%%                          %%%
%%% Содержание               %%%

\newpage

\tableofcontents

%%% Содержание              %%%
%%%                         %%%
%%%%%%%%%%%%%%%%%%%%%%%%%%%%%%%

%%%%%%%%%%%%%%%%%%%%%%%%%%%%%%%%
%%%                          %%%
%%% Введение                 %%%

%%% В введении Вы должны описать предметную область, с которой связана %%%
%%% Ваша работа, показать её актуальность, вкратце определить цель     %%%
%%% разработки					       %%%


\newpage
\section*{Введение}
\addcontentsline{toc}{section}{Введение}

Цель проекта:разработка компьютерной программы, реализующей игру «2048» на языке С++ и qml.


Задачи проекта: 
%%% Пример создания списков %%%
\begin{enumerate} 
    \item Проанализировать правила игры «2048», её оригинальную разработку, существующие аналоги. На основе проведенного анализа разработать концепцию требований к разрабатываемой программе.
    \item Разработать алгоритмы, необходимые для реализации игры «2048». Разработать макет пользовательского интерфейса.
    \item Разработать интерфейс и логику работы программы. Провести тестирование и отладку разработанной программы.
\end{enumerate}


%%% Пример добавления изображения %%%
%%% называйте изображение латиницей %%%

%%%                          %%%
%%%%%%%%%%%%%%%%%%%%%%%%%%%%%%%%

%%%%%%%%%%%%%%%%%%%%%%%%%%%%%%%
%%% Требования к приложению %%%
\newpage
\section{Требования к приложению}
% \subsection{Подраздел}
\begin{itemize}
  \item Логическая головоломка
  \item Неброский дизайн
  \item Простата использования
  \item Возможность запускать новую игру не перезапуская приложения
\end{itemize}



 
%%%                                     %%%
%%%%%%%%%%%%%%%%%%%%%%%%%%%%%%%%%%%%%%%%%%%

%%%%%%%%%%%%%%%%%%%%%%%%%%%%%%%%%%%%%%%%%%%
%%%                                     %%%
%%% Проектирование приложения           %%%
\newpage
\section{Проектирование приложения}
\begin{enumerate}
\item main.cpp-главный модуль для работы с функциями на языке <<С++>>
\item helper.cpp-основной файл для реализации следующих функций:
\begin{itemize}
\item newGame (Генерация новой игры)
\item right (Cмещение всех элементов вправо и если есть возможность сложение этих элементов)
\item left (Cмещение всех элементов влево и если есть возможность сложение этих элементов)
\item up (Cмещение всех элементов вверх и если есть возможность сложение этих элементов)
\item down (Cмещение всех элементов вниз и если есть возможность сложение этих элементов)
\end{itemize}
\item main.qml-Главный модуль графического интерфейса.
\item Brick.qml-Значение каждого из 16 квадратов.
\item Gameplay.qml-Графический интерфейс игрового поля.
\item Toolbar.qml-Графический интерфейс панели инструментов.
\end{enumerate}
%%%                          %%%
%%%%%%%%%%%%%%%%%%%%%%%%%%%%%%%%

%%%%%%%%%%%%%%%%%%%%%%%%%%%%%%%%
%%%                          %%%
%%% Реализация приложения    %%%
\newpage
\section{Реализация приложения}
Для реализации приложения были использованы языки <<C++>> и <<QML>>.
\begin{itemize}
\item Количество <<C++>> файлов: 2
\item Количество <<QML>> файлов: 4
\item Количество <<C++>> функций: 5
\item Количество строк <<C++>> кода: 189
\item Количество строк <<QML>> кода: 257
\end{itemize}


%%% Если необходимо вставками оформляются исключительно небольшие фрагменты кода.
%%% Для больших фрагментов используте приложение (пример после заключения)


%%%                          %%%
%%%%%%%%%%%%%%%%%%%%%%%%%%%%%%%%

%%%%%%%%%%%%%%%%%%%%%%%%%%%%%%%%
%%%                          %%%
%%% Заключение               %%%

\newpage
\section*{Заключение}
\addcontentsline{toc}{section}{Заключение}
В результате проекта было разработана игра 2048.
\\

Получен опыт работы с библиотеками языка <<С++>>, а также опыт работы с <<Qt Quick>>.

%%%                          %%%
%%%%%%%%%%%%%%%%%%%%%%%%%%%%%%%%

%%%%%%%%%%%%%%%%%%%%%%%%%%%%%%%%
%%%                          %%%
%%% Приложение               %%%

\newpage
\appendix
%\section*{Приложение}
%\addcontentsline{toc}{section}{Приложение}
%\titleformat{\section}[display]
%  {\normalfont\Large\bfseries}
%  {Приложение\ \thesection}
%  {0pt}{\Large\centering}
%\renewcommand{\thesection}{\Asbuk{section}}

%%% Ещё одно приложение
% \newpage
% \section*{Приложение Б.}
% \addcontentsline{toc}{section}{Приложение Б.}

%%%                          %%%
%%%%%%%%%%%%%%%%%%%%%%%%%%%%%%%%
\end{document}
