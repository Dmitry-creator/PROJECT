%%%%
% Преамбула: подключение необходимых пакетов
% Редактируйте осторожно!
%

\documentclass[hyperref={unicode}]{beamer}

\usepackage[utf8x]{inputenc}
\usepackage[english, russian]{babel}
\usepackage{color, colortbl}
\usepackage{rotating} 
\usepackage{graphicx}
\usepackage{algorithmic}

\usetheme[nosecheader]{PetrSU-CS}


%%%%
% Преамбула: основные параметры презентации
% Отредактируйте в соответствии с комментариями
%

\title[%
    % Краткое название работы не используется в этой презентации!
    % Бобры и Интернет
]{%
    % Полное название работы отображается на титульной странице
    Приложение <<Desktop crypto tool>>
}

% Подзаголовком опишите тип работы:
% - Курсовая работа
% - Выпускная квалификационная работа бакалавра
% - Дипломная работа
% - Магистерская диссертация
\subtitle{Отчет о проектной работе по курсу <<Основы информатики и программирования>>}

\author[%
    % Имя и фамилия автора работы отображаются на каждом слайде в нижнем колонтитуле
    Татьяна Квист
]{%
    % Имя, отчество и фамилия автора работы отображаются на титульном слайде
    Татьяна Денисовна Квист
}

\date[%
    % Дата защиты
    28.05.2021
]{%
    % Руководитель
    Научный руководитель: ст. преп., А. В. Бородин
}

\institute[%
    % Краткое название организации не используется в этой презентации
    ПетрГУ
]{%
    % Полное название организации и подразделения
    Петрозаводский государственный университет\\
    Кафедра информатики и математического обеспечения
}


%%%%
%
% Начало содержимого слайдов
%

\begin{document}

% Титульный слайд
\begin{frame}
\maketitle
\end{frame}

% Пример слайда для обоснования актуальности работы
\begin{frame}
  % Заголовок слайда
  \frametitle{Проблема автоматизации}
  В мире криптографии часто приходится решать какие-то задачи, которые человек может выполнять несколько часов, а то и дней. Для этого создаются приложения, автоматизирующие эти процессы: вычисление сложных математических операций, brute-force (метод перебора грубой силы) и т.п. Случаются ситуации, когда нет доступа к сети Интернет, в связи с этим создаются оффлайн приложения.
\end{frame}

% Пример слайда с формулировкой целей и задач
\begin{frame}
  % Заголовок слайда
  \frametitle{Цель и задачи}
  \begin{block}{Цель работы}
    Разработать приложение для решения основных криптографических задач.
  \end{block}
  \begin{block}{Задачи}
  \begin{itemize}
    \item разработать модуль для подбора строки, которой соответствует заданный хеш;
    \item разработать модуль для дешифровки RSA;
    \item разработать модуль для перевода кодировки из UTF-8 в base16, base32 и base64;
    \item разработать графический интерфейс пользователя;
    \item реализовать приложение с использованием разработанных модулей и QtQuick.
  \end{itemize}
  \end{block}
\end{frame}

\begin{frame}
    % Заголовок слайда
    \frametitle{Этапы разработки приложения}
    \begin{enumerate}
        \item Разработка модуля для подбора хеша.
        \item Разработка модуля для дешифровки RSA.
        \item Разработка перевода из одной кодировки в другие.
        \item Разработка графического интерфейса пользователя.
    \end{enumerate}
\end{frame}
  
\begin{frame}
    % Заголовок слайда
    \frametitle{Hash.cpp}
    Модуль, позволяющий подобрать строку, MD5-хеш которой будет соответствовать заданной строке. Подбор осуществляется методом грубой силы по словарю rockyou.txt. В классе содержится одна функция: check\_hash().
\end{frame}

\begin{frame}
    % Заголовок слайда
    \frametitle{RSA.cpp}
    Модуль, позволяющий дешифровать криптографический алгоритм RSA. Пользователь вводит переменные p, q, e и само зашифрованное сообщение. В классе содержится три функции:
    \begin{itemize}
        \item solve\_rsa() --- подготовка всех необходимых переменных для дешифровки RSA.
        \item calculateD() --- подсчёт числа d, для которого будет выполняться следующее условие: $d \cdot e = 1\; mod \; phi$
        \item decrypt() --- дешифровка сообщения со всеми необходимыми переменными.
    \end{itemize} 
\end{frame}

\begin{frame}
    % Заголовок слайда
    \frametitle{Bases.cpp}
    Модуль, позволяющий перевести заданную пользователем строку в base16, base32 и base64. Используется сторонняя библиотека для подсчёта base32 и base64: \url{https://github.com/tplgy/cppcodec}. Функция модуля: bases\_encode().
\end{frame}

% Пример заключительного слайда
\begin{frame}
  \frametitle{Заключение}
  
  Реализованные функции:
  
  \begin{itemize}
    \item Подбор MD5-хеша.
    \item Дешифровка RSA.
    \item Перевод строки из UTF-8 в base16 (hex), base32 и base64
  \end{itemize}
  
\end{frame}

\begin{frame}
    \frametitle{Заключение}
    
    В результате проекта было разработано приложение для решения основных криптографических задач.\\

    Предлагаемые дополнения для реализации: 
    \begin{itemize}
        \item определение кодировки для последующего перевода в другую кодировку;
        \item добавление шифров (например Цезарь, Атбаш и т.д.)
    \end{itemize}
    
\end{frame}
  
\begin{frame}
  \frametitle{}
  
{\Large\mbox{}\hfil Спасибо за внимание!}
  
\end{frame}
\end{document}
